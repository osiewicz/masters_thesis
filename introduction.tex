\chapter*{Wstęp}
\label{chap:wstep}
\addcontentsline{toc}{chapter}{Wstęp}
Wraz z rozwojem komputerów doszło do powstania grupy języków programowania wysokiego poziomu - oferujących wyższy poziom abstrakcji i łatwiejsze zrozumienie kodu źródłowego kosztem konieczności tłumaczenia  składni języka źródłowego na kod wykonywalny, czyli kompilacji.
Jakość programu wynikowego zależy między innymi od jakości przetłumaczonego kodu wykonywalnego - dlatego też współczesne kompilatory oprócz tłumaczenia odpowiadają też za optymalizację programu wynikowego. 
Przy ocenianiu jakości kodu maszynowego można przyjąć różne kryteria - wydajności, rozmiaru czy też poboru energii. 

Jakość kompilatora wykorzystywanego w procesie produkcji oprogramowania ma pośredni wpływ na pozycję firm na rynku. Wykorzystanie lepszego kompilatora oznacza niższe koszty utrzymania infrastruktury do budowy oprogramowania i wyższą jakość produktu finalnego, w związku z czym rozwój kompilatorów jest korzystny dla całej społeczności informatycznej. Głównym czynnikiem wpływającym na jakość kodu wykonywalnego oraz czas kompilacji jest agresywność zastosowanie metod optymalizacji.

Obecnie

\cite{EngineeringACompiler}
\section{Cel pracy}

\section{Plan pracy}
